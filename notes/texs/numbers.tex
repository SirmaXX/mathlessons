\documentclass[12pt]{article}
\usepackage[turkish]{babel} %Türkçe bölüm isimleri
\usepackage[utf8]{inputenc} %Türkçe karakterler
\usepackage{amssymb}
\begin{document}
\title{Matematik-Bilgisayar Bilimleri}
\author{Deniz Balcı\\
}
\renewcommand{\today}{Mayıs 31, 2019}
\maketitle


\section{Sayılar}

\subsection*{Sayıların türleri}

\begin{enumerate}
\item $\mathbb{P}$ Asal Sayılar( 2, 3, 5, 7, 11, 13, 17,) \\
\item $\mathbb{N}$ Doğal Sayılar(0 , 1 , 2 , 3 , 4 , ..)  \\
\item $\mathbb{W}$ Pozitif tam sayılar(1, 2, 3, 4, ..)\\
\item $\mathbb{Z}$ Tam sayılar(..,-2 , -1 , 0 , 1 , 2 ,..)\\
\item $\mathbb{Q}$ Rasyonel Sayılar ($\frac{1}{2}$,$\frac{2}{3}$,-$\frac{1}{12}$...)\\
\item $\mathbb{I}$ İrrasyonel Sayılar($\pi$,$\sqrt{0,72}$,$\frac{\sqrt{72}}{\sqrt{4}}$) \\
\item $\mathbb{R}$ Reel Sayılar  \\
\item $\mathbb{C}$ Karmaşık sayılar(2+3i,5-2i) \ldots 
\end{enumerate}


\end{document}