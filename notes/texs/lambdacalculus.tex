\documentclass[12pt]{article}


\usepackage[utf8]{inputenc} %Türkçe karakterler
\usepackage{amsmath} %matematik kütüphanesi
\usepackage{graphicx} %grafik kütüphanesi
\usepackage{hyperref} %url paketi

% yazar hakkında bilgiler
\title{Lambda Calculus}
\date{2019-08-31}
\author{Deniz Balcı}
% yazar hakkında bilgiler


% içerik hakkında bilgiler
\begin{document}
% sayfa numalandırma
  \pagenumbering{gobble}
  \maketitle
  
  
  \section{ ($|\lambda|$ Lambda Calculus nedir ?}
   Lambda kalkülüs ($lambda$ -Kalkülüs, Lambda Calculus) fonksiyonları soyutlama (abstraction), bağlama (binding), uygulama (application) ve ikame etme (substitution) kavramları üzerine kurulu bir hesaplama modelidir.Turing-church
\subsection{Lambda fonksiyonu nedir ?}
Lambda fonksiyonu ,Lambda kalkülüsten türetilmiş anonim fonksiyonlar yazmamızı sağlayan fonksiyondur.
\\
Örnek bilgisayar fonksiyonları
\\
\href{https://stackoverflow.com/questions/16501/what-is-a-lambda-function}{https://stackoverflow.com/questions/16501/what-is-a-lambda-function}

\section{Lambda Calculus Kuralları}
\subsection{Fonksiyon yapısı}
expr -> $lambda$ var.expr| expr expr| var (expr)
not:Bu ifade Context-Free Grammar ile oluşturulmuştur.Bir ifade lamba değikeni.ifade veya ifade ifade veya parantez içinde değişken olabilir.Not:sayıların mutlaka tam sayı olması gerekli.
\subsection{Lambda Fonksiyonu ile temel  Örnekler}
$|\lambda|$ x.(x+1) =2 bu durumda x =3 olur.\\
Bu ifadenin anlamı x değişkeni oluştur ($|\lambda|$x)\\
x ifadesine 1 ekle  (x =2 )\\
2+1 =3 \\
 şimdide bir sayının karesini alan bir fonksiyon oluşturalım.
 $|\lambda|$ x.(x*x)  bu ifadenin anlamı ise 
x değişkeni al kendisi ile çarp demektir.\\
Bu tip işlemleri 4 işlem ile yapabiliriz.(+ ,- , * ,/ ) şimdi ise 2 değişken kullanarak 2 değişkeni toplayan bir fonksiyon üretelim.\\
$|\lambda|$ x.$|\lambda|$ y (x+y) lambda x ve lambda y bizim değişkenlerimiz 2 farklı değişkene ait rakamların toplamını vermiş olduk.İstediğiniz kadar değişken ekleyebilirsiniz.

 
\subsection{Mantıksal lambda fonksiyonları }
Şimdi ise daha basit (ancak en karmaşık olabilen) doğru,yanlış veya değil kavramlarına bakalım. \\
TRUE =$|\lambda|$ x.$|\lambda|$ y .x  \\
FALSE=$|\lambda|$ x.$|\lambda|$ y .y  \\
NOT =$|\lambda|$ b.b.FALSE.TRUE  \\
x değişkenini doğru ve  y değişkenini yanlış olarak belirleyelim.1.fonksiyonda 2 x ve y adında 2 parametremiz mevcut ve bu 2 değeri alıp sadece 1.değeri yani x i döndüren fonksiyona sahibiz.\\
2.fonksiyonu incelediğimizde ise x ve y değerlerini alıp  y değerini döndürüyor yani yanlış  değeri döndürüyor.\\
3.fonksiyonu incelediğimizde ise girilen değerin değilini (yani zıttını çevirir) güzel bir örnek ile açıklayalım.\\
NOT TRUE = $|\lambda|$ b.b.FALSE.TRUE   ifademizi açalım.\\
NOT TRUE = $|\lambda|$ b.b. $|\lambda|$ x.$|\lambda|$ y .x. $|\lambda|$ x.$|\lambda|$ y .x\\
=TRUE
\\ Burada x değerini  denklemimiz şuna dönüşüyor.\\
$|\lambda|$ b.b  bu fonksiyonumuz  x değerini girdiğinde x değerini döndürür yani TRUE olur ve uygulamamız buna dönüşür\\
($|\lambda|$x.$|\lambda|$y.x.y )$|\lambda|$ x.$|\lambda|$ y .x parantez içindeki fonksiyon işe yaramaz ve sadece x yani true fonksiyonu kalır.sonuç true dur


  \newpage

\section{Kaynakça}
\begin{enumerate}
	\item  \href{https://www.youtube.com/watch?v=eis11j_iGMs}{Lambda Matematik - Computerphile}.
	\item  \href{https://www.youtube.com/watch?v=9l6YVf0xqIs}{Bilgisayar Kavramlari - Part 1}.
	\item  \href{https://www.youtube.com/watch?v=v0cNGm5YgZ8}{Bilgisayar Kavramlari - Part 2}.
	\item  \href{https://blog.lacriment.com/2019-02-01/fonskiyonel-programlama-1-lambda-kalkulus-1}{blog.lacriment.com}.
	\item  \href{https://stackoverflow.com/questions/16501/what-is-a-lambda-function}{what is lambda function}
	
	
\end{enumerate}





\end{document}